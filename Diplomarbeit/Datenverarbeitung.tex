\section{Datenverarbeitung}
\label{sec:Datenverarbeitung}
Der \ac{Raspi} ist für die Datenverarbeitung zuständig, einerseits empfängt er die Daten von den verschiedenen Sensoren, andererseits schreibt er alle gesammelten Messwerte auf die microSD-Karte. Diese Operationen werden alle von einem Python-Programm erledigt. Mithilfe eines Tasters kann zwischen zwei Modi umgeschalten werden, einer davon ist der Datenaufzeichnungsmodus, welcher durch eine eingeschaltete \ac{LED} indiziert wird. In diesem Modus werden die Daten von allen verbauten Sensoren gesammelt und anschließend auf den Speicher des \ac{Raspi} geschrieben. Jedes mal, wenn das Programm gestartet wird, wird eine neue Textdatei mithilfe der in Python integrierten Funktion \verb|open()| erstellt. Das von dieser Funktion zurückgegebene Objekt wird dann verwendet, um die erstellte Datei zu bearbeiten. Wenn ein neuer Datensatz gesammelt wird, wird dieser in eine neue Zeile in die Datei geschrieben. Jeder Datensatz besteht aus der Zeit der Datenaufnahme, der Orientierung, den Beschleunigungen, der Umgebungstemperatur, den Drehzahlen, der Geschwindigkeit und den Koordinaten. Die jeweiligen Werte sind voneinander mit Kommas getrennt, um die Kompatibilität mit Tabellenkalkulationsprogrammen wie Microsoft Excel oder LibreOffice Calc zu gewährleisten. Der andere zur Verfügung stehende Modus ist der der Datenübertragung auf ein \ac{USB}-Wechselmedium. In diesem Modus überprüft der Raspberry Pi jede Sekunde, ob an dem Einhängepunkt \verb|/media/usb0| ein Gerät erscheint. Sollte dies geschehen, werden alle erstellten Dateien in 100 Millisekunden-Abständen das Stammverzeichnis des \ac{USB}-Wechselmediums kopiert. Um einen erfolgreichen Kopiervorgang zu signalisieren, blinkt die eingebaute \ac{LED} pro kopierter Datei einmal. Dafür sollte der Datenträger mit einem Dateisystem wie \ac{NTFS} oder \ac{FAT} formatiert sein, welches sowohl unter Linux als auch unter Microsoft Windows eingehängt werden kann. 
\\
Zusätzlich zur Speicherung auf dem Raspberry Pi werden die Daten live in die Befehlszeile ausgegeben, in der das Programm ausgeführt wird. Somit können die Daten live auf einem Computer oder Smartphone ausgelesen werden, solange sich das Auto in Reichweite eines \ac{WLAN} Signals befindet und sich die Geräte im selben Netzwerk befinden. Um diese live-Übertragung zu verwenden, kann das Gerät, auf dem die Daten angezeigt werden sollen via \ac{SSH} mit dem Raspberry Pi verbunden werden und dann das Programm ausführen.