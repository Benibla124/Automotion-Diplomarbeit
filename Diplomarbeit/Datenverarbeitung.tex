\section{Datenverarbeitung}
\label{sec:Datenverarbeitung}
Der \ac{Raspi} ist für die Datenverarbeitung zuständig, einerseits empfängt er die Daten von den verschiedenen Sensoren, andererseits schreibt er alle gesammelten Messwerte auf die microSD-Karte. Diese Operationen werden alle von einem Python-Programm erledigt. Mithilfe eines Tasters kann zwischen zwei Modi umgeschalten werden, einer davon ist der Datenaufzeichnungsmodus. In diesem Modus werden die Daten von allen verbauten Sensoren gesammelt und anschließend auf den Speicher des \ac{Raspi} geschrieben. Jedes mal, wenn das Programm gestartet wird, wird eine neue Textdatei mithilfe der in Python integrierten Funktion \verb|open()| erstellt. Das von dieser Funktion zurückgegebene Objekt wird dann verwendet, um die erstellte Datei zu bearbeiten. Wenn ein neuer Datensatz gesammelt wird, wird dieser in eine neue Zeile in die Datei geschrieben. Jeder Datensatz besteht aus der Zeit der Datenaufnahme, der Orientierung, den Beschleunigungen, der Umgebungstemperatur, den Drehzahlen, der Geschwindigkeit und den Koordinaten.