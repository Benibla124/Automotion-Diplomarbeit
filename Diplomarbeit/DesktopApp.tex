\section{Visualisierung}
\label{sec:DesktopApp}
Für die Programmierung der Desktopanwendung wird ebenfalls Python verwendet, in diesem Fall wird zusätzlich auch Qt verwendet. Qt ist, wie in Sektion \ref{subsec:tQt} beschreiben, eine Bibliothek zur Entwicklung von grafischen Oberflächen, was eine Erleichterung in der Entwicklung von Desktopanwendungen darstellt. Die Desktopanwendung dieser Diplomarbeit ist in vier Ansichten aufgeteilt. Wird das Programm gestartet, wird zunächst die \glqq Overview\grqq angezeigt. In dieser Ansicht sind ein \ac{3D}-Modell des Autos und drei Schaltflächen zu sehen. Die Schaltflächen führen zu den anderen Ansichten, darunter die \glqq Table View\grqq , die \glqq Plot View\grqq \ und die \glqq Map View\grqq . Eine weitere Ansicht ist, wie bereits erwähnt, die \glqq Table View\grqq , in welcher eine Tabelle mit allen Messwerten der geöffneten Datei zu sehen ist.