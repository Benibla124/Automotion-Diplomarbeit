\section{Positionsbestimmung}
\label{sec:GPS}
Um die geografische Lage des Autos festzuhalten, wird ein \ac{GPS}-Modul verwendet. Das System\ac{GPS} wurde bereits in Sektion \ref{subsec:tGPS} erklärt. Dazu gehört die Wahl der Elektronik, die Montage sowie die programmatische Implementation.

\subsection{Wahl der GPS-Elektronik}
\label{subsec:GPSchoice}
Das GY-NEO6MV2 \ac{GPS}-Modul mit Keramikantenne bietet eine günstige Möglichkeit zur Positionsbestimmung. Mit einer Genauigkeit von 1,5 bis 6 Metern, liefert es die Positionsdaten über die \ac{UART}-Schnittstelle, mithilfe des \ac{NMEA}-Protokolls. Näheres zur Datenübertragung wird in Sektion \ref{subsec:GPSprogram} erklärt. Die Wiederholungsrate der Datenpakete liegt standardmäßig bei 1Hz und kann auf bis zu 5Hz erhöht werden. Die Standard-Baudrate über die serielle Schnittstelle beträgt 9600 b/s. Versorgt wird das GY-NEO6MV2 mit 2.7 bis 5 V und arbeitet in einem Temperaturbereich von -40°C bis 85°C. Es hat eine Kaltstart-Zeit von 27 Sekunden und eine Hotstart-Zeit von 1 Sekunde. Ein Kaltstart ist bei einem GPS-Gerät die erste Erfassung von Satelliten-Bahnen und Zeiten bei der ersten Aktivierung, oder wenn sich das Gerät von der letzten Ortsbestimmung weit entfernt hat. Ein Hotstart tritt bei einer Aktivierung des GPS-Moduls dann auf, wenn es die Satelliten-Daten noch weiß. Das ist meistens der Fall, wenn nicht mehr als sechs Stunden seit der letzten Positionserfassung vergangen sind und sich das Gerät seit dem nicht weit entfernt hat. Die angegebenen Zeiten sind dabei die kleinstmöglichen. Die dazugehörige Keramikantenne wird über eine UFL-Hirose-Verbindung am GPS-Modul angesteckt. 

\subsection{Montage der GPS-Elektronik}
\label{subsec:GPSmount}
Das GY-NEO6MV2 Modul ist an der Innenseite des Deckels des Elektronikgehäuses mit doppelseitigem Klebeband angebracht. Der Vorteil dieser Montagestelle ist die Fixiermöglichkeit der Antenne. Diese muss, um Satelliten-Daten empfangen zu können in den freien Himmel zeigen. Über eine Bohrung im Deckel des Gehäuses der Elektronik ist die Antenne mit der Platine verbunden.

\subsection{Programmatische Implementation}
\label{subsec:GPSprogram}
Wie bereits erwähnt liefert das GPS-Modul die Daten über die UART-Schnittstelle. Dabei wird die transmit (TX)-Leitung des GPS-Moduls mit dem receive (RX)-Pin des RaspberryPi verbunden. Die receive-Leitung des Moduls ist nicht nötig, da es nur die Daten an den RaspberryPi senden soll.\\
Um den seriellen (UART) Port des RaspberryPi verwenden zu können muss dieser zunächst in der Konfigurationsdatei  \verb|/boot/config.txt| mit dem Hinzufügen der Zeile \verb|enable_uart=1| aktiviert werden. Des Weiteren muss die Konsole \glqq getty\grqq\ der seriellen Schnittstelle \verb| /dev/ttyAMA0| deaktiviert werden, um diese in einem Python Programm einbinden zu können. Dies erfolgt mit dem Befehl \verb|sudo systemctl stop serial-getty@ttyAMA0.service|\ und dem Befehl \verb|sudo systemctl|\\
\verb|disable serial-getty@ttyAMA0.service|.\\
Das \ac{GPS}-Modul gibt die Daten mit dem \ac{NMEA}-Protokoll an den RaspberryPi weiter. Der Inhalt eines Datenpaketes in diesem Format sieht wie in Abbildung \ref{fig:NMEA} aus. 
