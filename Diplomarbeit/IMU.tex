\section{Inertiale Messeinheit}
\label{sec:IMU}
Wie in Sektion \ref{subsec:tIMU} beschrieben, handelt es sich hierbei um eine Kombination aus Accelerometern, Gyroskopen und Magnetometern, die zur Bestimmung von Beschleunigungen und Lage im dreidimensionalen Raum verwendet wird.
\subsection{Wahl der inertialen Messeinheit}
\label{subsec:IMUchoice}

\subsection{Montage der inertialen Messeinheit}
\label{subsec:IMUmount}

\subsection{Programmatische Implementation der inertialen Messeinheit}
\label{subsec:IMUprogram}
Der MPU9250 stellt die gemessenen Werte über den \ac{I2C} Bus bereit. Um diese Werte mit Python am Raspberry Pi auslesen zu können, kann die Bibliothek \glqq smbus\grqq \ (\url{https://pypi.org/project/smbus/}) verwendet werden. Smbus ist eine Python-Implementation der i2c-tools für den Linux-Kernel. Mithilfe von smbus können die gemessenen Werte aus den jeweiligen Registern entnommen werden. Diese Werte enthalten aber die Erdbeschleunigung und stellen noch keine Lage dar. Um eine zuverlässige Bestimmung der Lage im dreidimensionalen Raum zu ermöglichen, muss wie in Sektion \ref{subsec:tIMU} beschrieben Sensorfusion betrieben werden. Für die einfache Implementation dieses Prozesses gibt es die Python-Bibliothek \glqq imusensor\grqq \ (\url{https://pypi.org/project/imusensor/}). Diese Bibliothek bietet zwei verschiedene Filter für die Sensorfusion an, Kalman und Madgwick. Während Kalman simpel ist und nicht sehr viele Prozessorressourcen benötigt, stellt Madgwick glattere Werte bereit, benötigt dafür aber mehr Rechenaufwand. Da der Raspberry Pi diesen Rechenaufwand problemlos aufbringen kann, wird Madgwick verwendet. \\
Diese Werte beinhalten aber noch immer die Erdanziehungskraft, was in Fällen, in denen sich das ferngesteuerte Auto nicht in einer horizontalen Lage befindet die Messwerte in x- und y-Richtung beeinflusst. Um dieses Problem zu lösen kann mithilfe der Lage im Raum bestimmt werden, auf welche Achse welcher Anteil der Erdbeschleunigung  wirkt. Die dafür benötigten Formeln können aus den Winkelfunktionen abgeleitet werden.
\begin{equation}
Fx=\frac{g*\tan(\gamma)}{\sqrt{\frac{\tan(\gamma)^2}{\cos(\beta)^2}+1}}
\end{equation}

\begin{equation}
Fy=\frac{g}{\sqrt{\tan(\gamma)^2+\tan(\alpha)^2+1}}
\end{equation}

\begin{equation}
Fz=\frac{g}{\sqrt{\cot(\alpha)^2+\cot(\beta)^2+1}}
\end{equation}

Dabei stellt g die Erdbeschleunigung, $\alpha$ die Rotation um die x-Achse (roll), $\beta$ die Rotation um die y-Achse (pitch) und $\gamma$ die Rotation um die z-Achse (yaw) dar.