\section{Drehzahlmessung}
\label{sec:RPM}
Die individuelle Drehzahl aller vier Räder des Modellautos soll erfasst werden. Dafür sind die Auswahl verschiedener Sensortechnologien, die Konstruktion, Fertigung und Montage der Halterungen sowie die Programmierung des Einplatinencomputers notwendig.

\subsection{Wahl der Sensortechnologie}
\label{subsec:RPMchoice}
Aufgrund des Hinterradantriebes des Modellautos sind für die vorderen und hinteren Räder unterschiedliche Sensortypen erforderlich. Durch die Federung des Fahrgestells und der Lenkung der Vorderräder, kommen einige Herausforderungen bei der Sensorwahl und Montage auf. Die Drehzahlerfassung der Hinterräder erfolgt deswegen direkt an den Radantriebswellen des Differentials. Bei den Vorderrädern muss die Drehzahl an den Rädern gemessen werden, da diese nur mitlaufen und keine Welle aufweisen.\\
\subsubsection{Sensortechnologie für Hinterräder}
\label{subsubsec:RPMchoiceRear}
Für die Drehzahlmessung der Hinterräder werden Infrarot-Gabellichtschranken, wie in Sektion \ref{subsec:tIR} beschrieben, verwendet. Diese geben die Versorgungsspannung von +3.3 V aus, wenn der \ac{IR}-Strahl unterbrochen wird. Auf der zu messenden Welle ist folglich eine Rotationsscheibe anzubringen, welche den \ac{IR}-Strahl periodisch unterbricht. Über die Zeit, die zwischen den einzelnen Unterbrechungen vergeht, lässt sich dann die Drehzahl ermitteln.\\
\subsubsection{Sensortechnologie für Vorderräder}
\label{subsubsec:RPMchoiceFront}
Wie bereits erwähnt ist es nur möglich, die Drehzahl der Vorderräder direkt am Rad zu erfassen. Eine Möglichkeit dafür wäre ebenfalls \ac{IR}-Sensoren zu verwenden. Hier wäre eine Ausführung mit parallel platziertem Sender und Empfänger, anstatt gegenüberliegend, notwendig. Auf diese Art und Weise kann eine schwarze Scheibe am Rad fixiert werden, welche weiße Segmente aufweist. Diese Segmente reflektieren den \ac{IR}-Strahl bei jeder Vorbeibewegung am Sensor. Dadurch kann wieder über die Zeit, die zwischen den Reflexionen vergeht, die Drehzahl ermittelt werden. Unter Berücksichtigung der Nachteile von \ac{IR}-Sensoren und dem Einbauort an den Vorderrädern, treten mehrere potentielle Probleme auf. Zum einen wäre es möglich, dass das Sonnenlicht mit dessen \ac{IR}-Anteil, vom Empfänger des Sensors fälschlicherweise erkannt wird. Bei den Vorderrädern ist die Montageposition relativ offen und ein direkter Einfall von Außenlicht wäre dadurch möglich. Aufgrund dieser ungeschützten Lage, wäre es auch denkbar, dass zum Beispiel Staub oder feiner Sand am Sender, beziehungsweise Empfänger, des \ac{IR}-Sensors haften bleibt und diesen funktionsunfähig macht.\\
Eine robuste und zuverlässige Lösung für die Drehzahlerfassung der Vorderräder sind digitale Hall-Sensoren, welche in Sektion \ref{subsec:tHall} bereits erklärt wurden. Um diese implementieren zu können wird ebenfalls eine Scheibe an der Innenseite der Räder montiert, welche allerdings Permanentmagneten aufweist.



\subsection{Montage der Drehzahlsensoren}
Auf der Differentialwelle ist ein Ring mit vier länglich ausgeführten Extrusionen angebracht, welche den Strahl pro Umdrehung der Welle vier mal unterbrechen.
