\section{Verwaltung des Programmcodes}
\label{sec:CodeVerwaltung}
Die Verwaltung des Programmcodes geschieht über das in Sektion \ref{subsec:tGit} beschriebenen Versionsverwaltungssystem Git, als Hostanbieter für die remote-Repositories wird Microsoft GitHub verwendet. Es werden drei verschiedene Repositories verwendet, eines für den in Sektion \ref{sec:Datenverarbeitung} beschriebenen Programmcode des Raspberry Pi, eines für den Programmcode der in Sektion \ref{sec:DesktopApp} beschriebenen Desktopanwendung und eines für den \hologo{LaTeX}-Quellcode dieses Dokumentes und der Abschlusspräsentation. Für Neuimplementationen von Funktionen am Raspberry Pi wird jeweils ein neuer Branch erstellt, nachdem die Funktion getestet ist, wird dieser Branch mit dem main-Branch zusammengeführt. Kleine Änderungen können auch direkt im Browser vorgenommen werden, GitHub bietet dafür einen eigenen online Editor an. Der Vorteil von Git für dieses Projekt ist es, dass beide Diplomanden gleichzeitig programmieren können, ohne, dass Konflikte im Code entstehen. Sollten trotzdem Probleme mit einer Version des Programmcodes auftreten, kann jederzeit eine ältere Version wiederhergestellt werden. 