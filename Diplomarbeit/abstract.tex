\begin{otherlanguage}{english} 
\begin{abstract}
\noindent
Various forces and moments act on cars while driving. Although these variables are measured in real cars, they are not made directly accessible to the driver. Instead, they are used for various systems such as ABS or traction control. In order to make these parameters evaluable, a remote-controlled model car is to be equipped with sensor technology. Factors relevant to driving dynamics include accelerations as well as the orientation in three-dimensional space and the speeds of the four wheels. Furthermore, the location of the vehicle is to be recorded with the help of the \ac{GPS}. Separate housings and mountings are to be made for the installation of the electronics. A single-board computer from Raspberry Pi is being used to enable the storage of data on the remote-controlled car. It is also important that the collected data can be analysed and displayed with a user-friendly graphical interface. \\
This diploma thesis was carried out in the course of the standardised diploma examination at the HTL Kapfenberg as well as financed by the HTL. In the future, this diploma thesis will be used for educational purposes.
\end{abstract}
\end{otherlanguage}
\newpage
\begin{abstract}
\noindent
Auf Autos wirken während der Fahrt diverse Kräfte und Momente. Diese Größen werden bei realen Autos zwar gemessen, für den Fahrer aber nicht direkt zugänglich gemacht. Stattdessen werden sie für verschiedene Systeme wie ABS oder Traktionskontrolle verwendet. Um diese Kenngrößen auswertbar zu machen, soll ein ferngesteuertes Modellauto mit Sensorik ausgestattet werden. Für die Fahrdynamik relevante Faktoren sind unter anderem die Beschleunigungen sowie die Lage im dreidimensionalen Raum und die Drehzahlen der vier Räder. Weiters ist der Standort des Fahrzeuges mithilfe des \ac{GPS} aufzuzeichnen. Für die Unterbringung und Montage der Elektronik sind eigene Gehäuse sowie Befestigungen anzufertigen. Um die korrekte Speicherung der Daten am ferngesteuerten Auto zu ermöglichen wird ein Einplatinencomputer von Raspberry Pi verwendet. Außerdem ist es wichtig, dass die erfassten Daten mit einer benutzerfreundlichen grafischen Oberfläche ausgewertet und dargestellt werden können. \\
Diese Diplomarbeit wurde im Zuge der standardisierten Reife- und Diplomprüfung an der HTL Kapfenberg durchgeführt sowie von dieser finanziert. In Zukunft soll diese Arbeit für Lehrzwecke verwendet werden.
\end{abstract}