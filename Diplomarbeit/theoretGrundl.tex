\section{Theoretische Grundlagen}
\label{sec:theoretGrundl}
Dieses Kapitel behandelt die theoretischen Grundlagen für verwendete Verfahren, Hardware und Software. Es hat die Aufgabe den Leser:innen für das Verständnis essentielle Informationen zu übermitteln.
\subsection{Globales Positionsbestimmungssystem}
\label{subsec:tGPS}
%TODO rewrite this
\glqq GPS ist ein satellitengestütztes Navigationssystem mit 32 aktiven Satelliten (Stand: 2014) und heißt vollständig "Navigational Satellite Timing and Ranging - Global Positioning System" (NAVSTAR-GPS). Das Verteidigungsministerium der USA (Department of Defense, DOD) ist der Betreiber von GPS und hat es hauptsächlich für militärische Zwecke vorgesehen. GPS wurde am 17. Juli 1995 offiziell in Betrieb genommen.

Mit GPS lässt sich der Standort und die Zeit ermitteln. Die Standortbestimmung umfasst die geographische Länge und Breite, sowie die Höhe. Die Genauigkeit liegt zwischen 13 Meter und 1 Millimeter. Genauigkeiten unter 2 Meter sind nur mit viel Aufwand (Differential-GPS) erreichbar. Übliche GPS-Empfänger und Handys haben eine Genauigkeit von 13 bis 2 Meter. Die horizontale Genauigkeit liegt bei 10 bis 15 Meter.
Drei Satelliten werden benötigt, um Längen-, Breitengrad und Höhe errechnen zu können. Der vierte Satellit wird benötigt, um die Uhr des Empfängers mit den Uhren der Satelliten synchronisieren zu können. Die Zeiten müssen absolut übereinstimmen, sonst kommt es zur Fehlberechnung.
Einfach ausgedrückt, wird aus der Laufzeit des Signals vom Satelliten die Entfernung zum Satelliten vom Empfänger errechnet. In Kombination mit mehreren Satelliten kann man die Position auf der Erdoberfläche bestimmen (Triangulation).\grqq \ 
(\cite{schnabelGPS})

\subsection{Inertiale Messeinheit}
\label{subsec:tIMU}
Eine \ac{IMU} z. Dt. inertiale Messeinheit kombiniert mehrere verschiedene Sensoren zur Bestimmung von Lage, Beschleunigungen und Position im dreidimensionalen Raum. Um eine zuverlässige Ermittlung dieser Werte zu gewährleisten, werden Beschleunigungsmessung, Rotationsgeschwindigkeitsmessung (und in einigen Fällen auch die Messung des Magnetfeldes der Erde) kombiniert. Inertiale Messeinheiten werden unter anderem für die Navigation von Flugzeugen, Raumschiffen und Schiffen verwendet. In der Automobilindustrie werden \ac{IMU}s verwendet, um das Fahrverhalten von Autos zu bestimmen und dieses zu verbessern.\\
Die \ac{IMU} liefert die Beschleunigungen sowohl in x-, y- als auch z-Richtung, die Winkelgeschwindigkeiten um die x-, y- als auch z-Achse (und in manchen Fällen die magnetische Feldstärke in x-, y- als auch z-Richtung). Um die Lage zu bestimmen können die Beschleunigungssensoren verwendet werden, indem die Lage der Erdbeschleunigung im dreidimensionalen Raum ermittelt wird, dies funktioniert allerdings nur, solange der Sensor sich in Ruhelage befindet. Alternativ können die Rotationsgeschwindigkeiten integriert werden, dies ergibt aber aufgrund von Messfehlern nach einiger Zeit Abweichungen (drift).\\ Um eine fehlerfreie Ermittlung der Lage zu ermöglichen kann Sensorfusion betrieben werden. Darunter versteht man einen Prozess, der Signale von zwei oder mehr Sensoren zusammenfasst. Im Falle einer \ac{IMU} kann die Ermittlung mittels Accelerometern, Gyroskopen und Magnetometern wie in Sektion \ref{subsec:IMUprogram} beschrieben kombiniert werden.
(\cite{UCAM-CL-TR-696})

\subsection{Python}
\label{subsec:tPython}
Python ist eine in den frühen 1990ern von Guido van Rossum erstellte Programmiersprache. Heutzutage zeichnet sich Python dadurch aus, dass es quelloffen und für jeden nutzbar ist. Außerdem nimmt Python dem Programmierenden durch die im Vergleich zu Sprachen wie C++ einfache Syntax und das integrierte Ressourcenmanagement viel Arbeit ab. Ein Nachteil von Python ist, dass es im Vergleich mit anderen Sprache wie C++ eine sehr langsame Programmiersprache ist. Aufgrund seiner Simplizität und Verfügbarkeit für Plattformen wie Microsoft Windows, MacOS, Linux und FreeBSD gibt es sehr viele quelloffene Bibliotheken für Python, was das Programmieren noch weiter erleichtert. Ein weiterer Unterschied zu Sprachen wie C, C++ oder Rust besteht darin, dass Python interpretiert statt kompiliert wird. Dies hat zur Folge, dass der Code direkt ausgeführt werden kann, ohne zuvor für jede Plattform eigens kompiliert werden zu müssen.
(\cite{matthes-2019})